\section*{Appendix}
\label{appx:proof}

\revise{Appendices are supporting material that has not been peer-reviewed.}

\parab{Correctness Analysis.}
\revise{In reliable \sys{}, a scattering $M$ from sender $S$ to receivers $R_i$ with timestamp $T$ should be delivered if and only if the $S$ does not fail before $T$ and $S$ has received all ACKs from $R_i$. If $S$ fails before $T$ (\emph{i.e.}, the failure timestamp of $S$ is less than $T$), then $M$ is not committed (or the commit message has not propagated to any receiver, which is regarded as not committed). If any receiver $R_i$ fails before receiving the prepare message of $M$ or sending the ACK message, $S$ should recall $M$.}

\revise{Now, we analyze the behavior of message $M: S \rightarrow R$ under each type of failure. A process has both sender and receiver roles, and they are considered separately.}
\begin{ecompact}
	\item Packet loss on a fair-loss link: deliver because 2PC retransmits packets.
	\item $R$ fails before sending ACK in Prepare phase: discard according to Recall step.
	\item $R$ fails after sending ACK in Prepare phase: \revise{The message is in $R$'s receive buffer. We only ensure atomicity in a fail-recover model because recording the last message that $R$ has delivered is impossible. The controller records failure timestamps of senders and undeliverable recall messages, so that recovered receivers can deliver or discard messages consistently in each scattering.} If $R$ fails permanently, atomicity is violated.
	\item $S$ fails before sending commit timestamp $T$: discard according to Discard step. \revise{For example, when $S$ fails while executing Recall step of $M$ due to previous failure of a receiver, its commit timestamp $T$ must be lower than $M$'s timestamp, so all receivers would discard $M$ due to failure of $S$.}
	\item $S$ fails after sending commit timestamp $T$: deliver if and only if the failure timestamp $T'$ of $S$ is larger or equal to $T$. Failure timestamp is determined by controller, which ensures that $S$ has committed $T'$ but no message after $T'$ is delivered. Controller obtains $T'$ by gathering the maximum failure timestamp from a cut \finalrevise{in a routing graph} consisting of healthy switches that separates $S$ and all receivers. \revise{If the data center has separate production and management networks, and assuming} they do not fail \finalrevise{simultaneously}, such a cut can always be found. Otherwise, \revise{a network partition may} lead to atomicity violation.
	\item The receiver $R'$ of another message $M'$ in the same scattering fails before sending ACK in Prepare phase: because the ACK is not received, discard according to Recall step.
	\item The receiver $R'$ of another message $M'$ in the same scattering fails after sending ACK in Prepare phase: deliver according to 2PC.
	\item The network path between $S$ and $R$ fails (\textit{e.g.}, due to routing problem), but $S$ and $R$ are reachable from the controller: controller forwards messages between $S$ and $R$.
	\item The network path between $S$ and $R$ fails, and $S$ or $R$ is unreachable from the controller: the unreachable process is considered as failed. For example, if a host or the only network link from a host fails, all processes on it are disconnected from the network, so they are considered to fail.
	%If a Top-of-Rack (ToR) switch fails, and hosts in the rack are only connected to one switch, all processes in the rack are considered to fail.
	\item The network path to $R$ fails after $R$ receives commit barrier $T$: $T$ is already delivered to $R$.
	\item The network path to $R$ fails before $R$ receives commit barrier $T$: deliver after $R$ recovers according to Receiver Recovery. If $R$ fails permanently, atomicity is violated.
\end{ecompact}

\iffalse
\newtheorem{theorem}{Theorem}
\theoremstyle{definition}
\newtheorem{definition}{Definition}
\newtheorem{property}{Property}
\newtheorem*{assump}{Assumption}
\newtheorem*{notation}{Notation}
\newtheorem{lemma}{Lemma}
\theoremstyle{remark}
\newtheorem*{rem}{Remark}
\newtheorem*{conclusion}{Conclusion}
\newtheorem*{note}{Note}
\subsection{Proof of hierarchical merge}
\label{appx:hierarchical_merge}
In Sec.\ref{sec:ideal}, we have pointed out the property of \textit{barrier timestamp} and how we aggregate them hierarchically.
Here we prove that the hierarchical aggregation does maintain the property.
\begin{property}\label{prop:barrier}
	Take any network link $L$, if a packet with barrier timestamp $B$ is on the $L$, then for any following packet carrying (barrier or data) timestamp $T$, there must be $B \le T$.
	
	When a switch or a host receives a packet with barrier timestamp $B$ from a network link $L$, it indicates that the message timestamp and barrier timestamp of future arrival packets from link $L$ will be larger than $B$.
\end{property}
\begin{theorem}
	If we derive barriers following (\ref{equ:derive_barriers}), then the barrier timestamp $B_{new}$ determined for output link $o$ still has the above property as long as the barrier from every input link has.
\end{theorem}
\begin{proof}
	We reuse notation $\mathcal{I}_o$ and $R_i$ defined in (\ref{equ:derive_barriers}).
	After barrier timestamp $B_{new}$ is sent through output link $o$, let's take  any following packets through $o$ and assume it comes from input link $i \in \mathcal{I}$, with \textit{message timestamp} $T$ unmodified and \textit{barrier timestamp} updated to $B'_{new}$.
	
	Because all packets coming through $i$ also have the Property~\ref{prop:barrier}, we have $B_{new} \le R_i \le T$.
	
	As for the barrier timestamp $B'_{new}$, let's assume it is determined by newer register value $R'_i$.
	$$B'_{new} := \min\{R'_i|i \in \mathcal{I}_o\} = R'_I$$
	
	Because of Property~\ref{prop:barrier}, we have $\forall i \in \mathcal{I}_o:R_i \le R'_i$, which implies $B_{new} \le R_I \le R'_I = B'_{new}$.

	From $B_{new} \le T$ and $B_{new} \le B'_{new}$, we can conclude that the $B_{new}$ on link $o$ still has Property~\ref{prop:barrier}.
\end{proof}

Besides, an endhost sender assigns barrier timestamps to be the same as data timestamps, which follows Property~\ref{prop:barrier} trivially, so hierarchical merge can maintain barrier's property properly.
\subsection{Proof of Minimax Synchronization}
\label{appx:minimax}

\begin{notation}
	Each sender $i\in\mathcal{S}$ has a logical timestamp $T_i$, which consists of physical clock $Phy$, physical clock skew $s_i$ and logical timestamp offset $O_i$ ($T_i = Phy+s_i+O_i$).
	Each receiver $j \in \mathcal{R}$ also has physical clock skew $s_j$.
\end{notation}\begin{notation}
	One-way delay from sender $i \in \mathcal{S}$ to receiver $j \in \mathcal{R}$ is $d_{ij}$.
	$d_{ij}$ is unmeasurable.
\end{notation}
\begin{notation}
	One-way delay from receiver $j \in \mathcal{R}$ to sender $i \in \mathcal{S}$ is $r_{ji}$.
	$r_{ji}$ is unmeasurable.
\end{notation}
\begin{notation}
	Round trip delay between sender $i \in \mathcal{S}$ and receiver $j \in \mathcal{R}$ is $RTT_{ij}( = RTT_{ji} = d_{ij} + r_{ji})$.
	$RTT_{ij}$ is measurable.
\end{notation}
\begin{note}
	A common endhost functions as both sender and receiver.
	So the delay between its two parts is $0$.
\end{note}
\begin{assump}
	There are no switches.
\end{assump}
\begin{assump}
	Senders and receivers are fully connected.
	The delay between physically unconnected sender and receiver is $+\infty$.
\end{assump}
\begin{assump}
	Delay is invariant during the synchronization.
\end{assump}
\begin{lemma}\label{proof:ts2offset}
	In the above model, logical timestamp offset is equivalent to logical timestamp.
\end{lemma}
\begin{proof}
	With linear time assumption, receiver $j$ will receive timestamp $(Phy+O_i-d_{ij}+s_i-s_j)$ from sender $i$ at physical time $Phy$.
	The receiver to sender direction is similar.
	We can count clock skew into link delay, which will make logical timestamp to only consist of physical time, timestamp offset and link delay.
	Besides, $Phy$ is the same for every sender and receiver, so we can treat logical timestamp offset $O_i$ equivalent to logical timestamp $T_i$.
\end{proof}
Now we prove several good mathematical properties assuming $\texttt{FwdFunc} =$ \textbf{max} and $\texttt{BackFunc} =$ \textbf{min}.
\theoremstyle{definition}
\begin{definition}{Let $O'_i$ be the new value of $O_i$}\label{proof:NextO_i}
\begin{equation*}
\begin{aligned}
O_i' &= \min\limits_{\forall j \in \mathcal{R}}\left[\max\limits_{\forall k \in \mathcal{S}}\left(O_k - d_{kj}\right) - r_{ji} + RTT_{ij}\right] \\
     &= \min\limits_{\forall j \in \mathcal{R}}\left[\max\limits_{\forall k \in \mathcal{S}}\left(O_k - d_{kj}\right) + d_{ij}\right] \\
     &=\min_{\forall  j \in \mathcal{R}}\left[\max_{\forall k \in \mathcal{S}}\left(O_k - d_{kj} + d_{ij}\right)\right]
\end{aligned}
\end{equation*}
\end{definition}
\newcommand{\FM}[1]{{\mathcal{M}_{#1}}}
\newcommand{\FMGEN}[2]{{O_{#1}-d_{{#1}{#2}}}}
\newcommand{\Fm}[1]{{m_{#1}}}
\newcommand{\FmGEN}[2]{\FMGEN{\FM{#1}}{#1}+ d_{{#2}{#1}}}
\newcommand{\NEXTT}[1]{{\FmGEN{\Fm{#1}}{#1}}}
\begin{definition} $\FM{j} \in \mathcal{S} (j \in \mathcal{R})$
\begin{equation} \label{proof:FMdef}
\forall j \in \mathcal{R}, \forall k \in \mathcal{S}:\FMGEN{\FM{j}}{j} \ge \FMGEN{k}{j}
\end{equation}
\end{definition}
\begin{definition} $\Fm{i} \in \mathcal{R} (i \in \mathcal{S})$
\begin{multline} \label{proof:Fmdef}
\forall i \in \mathcal{S}, \forall j \in \mathcal{R}:\\
(\FmGEN{\Fm{i}}{i}) \le (\FmGEN{j}{i})
\end{multline}
\end{definition}
\begin{note}
	The Definition~\ref{proof:NextO_i} can then be rewritten as
	\begin{equation}\label{proof:NextO_i2}
	O_i' \triangleq \NEXTT{i}
	\end{equation}
\end{note}
\begin{note}
	From (\ref{proof:Fmdef}) and (\ref{proof:NextO_i2}), we have
	\begin{equation}\label{proof:FMnewOle}
		\forall j \in \mathcal{R} : O'_i - d_{ij} \le \FMGEN{\FM{j}}{j}
	\end{equation}
	
	Which means
	\begin{equation}\label{proof:unchanged_FM}
		\forall j \in \mathcal{R}:O'_{\FM{j}} = O_{\FM{j}}
	\end{equation}
\end{note}
\begin{theorem}\label{thm:someO_unchanged}
	$\exists i \in \mathcal{S}:O'_i=O_i$
\end{theorem}
\begin{proof}
Let $k=i$ in (\ref{proof:FMdef}), we have
\begin{equation*}\begin{split}
\forall i \in \mathcal{S}, \forall j \in \mathcal{R} : \FMGEN{\FM{j}}{j} \ge O_i - d_{ij}\\
\Leftrightarrow\hspace{1em} \FmGEN{j}{i} \ge O_i
\end{split}\end{equation*}
With (\ref{proof:NextO_i2}), we have
\begin{equation} \label{proof:newOge}
\forall i \in \mathcal{S}: O'_i \ge O_i
\end{equation}
With (\ref{proof:Fmdef}) and (\ref{proof:NextO_i2}), we have
\begin{equation} \label{proof:newOle}
\forall i \in \{i \in \mathcal{S}| \exists j \in \mathcal{R}:\FM{j}=i\}:O'_i \le O_i
\end{equation}
With (\ref{proof:newOge}) and (\ref{proof:newOle}), we conclude
\begin{equation}\label{proof:O_i_unchanged}
\forall i \in \{i \in \mathcal{S}| \exists j \in \mathcal{R}:\FM{j}=i\}:O'_i = O_i
\end{equation}
\end{proof}
\begin{theorem}\label{thm:allnewO_unchanged}
	$\forall i \in \mathcal{S}: O''_i = O'_i$, where $O''_i$ is the new value of $O'_i$
\end{theorem}
\begin{proof}
From (\ref{proof:Fmdef}) and (\ref{proof:NextO_i2}), let $m_i = j$, we have

\begin{equation}\label{proof:newnewO1}
	\forall i,k \in \mathcal{S}: O'_k\le(\FmGEN{\Fm{i}}{k})
\end{equation}
Substitute the definition of $O_i'$ from (\ref{proof:NextO_i2}), we have
\begin{equation}\label{proof:newnewO2}
	\forall i,k \in \mathcal{S}: O'_k\le (O'_i + d_{k\Fm{i}} - d_{i\Fm{i}})
\end{equation}
From (\ref{proof:newnewO1}) and (\ref{proof:newnewO2}), we have
\begin{equation*}
	\forall i \in \mathcal{S}, \exists j \in \mathcal{R}, \forall k \in \mathcal{S}: (O'_k-d_{kj}+d_{ij}) \le O'_i
\end{equation*}
So by the Definition~\ref{proof:NextO_i}
\begin{equation} \label{proof:newnewOle}
O''_i = \min_{\forall  j \in \mathcal{R}}\left[\max_{\forall k \in \mathcal{S}}\left(O'_k - d_{kj} + d_{ij}\right)\right]\le O'_i
\end{equation}
Similar to (\ref{proof:newOge}), we can prove
\begin{equation}\label{proof:newnewOge}
	\forall i \in \mathcal{S}:O''_i \ge O'_i
\end{equation}
From (\ref{proof:newnewOle}) and (\ref{proof:newnewOge}), we have $\forall i \in \mathcal{S}: O''_i=O'_i$
\end{proof}
\begin{definition}
	The reorder delay $D$ of overall system is the maximum timestamp skew among all receivers.
	\begin{equation*}
		D:=\max_{\substack{\forall j \in \mathcal{R}\\\forall k \in \mathcal{S}}}\left[\left(\FMGEN{\FM{j}}{j}\right)-\left(\FMGEN{k}{j}\right)\right]
	\end{equation*}
\end{definition}
\begin{theorem}\label{thm:reorder_delay_decline}
	With arbitrary initial $O_i(i\in \mathcal{S})$, the initial reorder delay $D$ $\ge$ the reorder delay $D'$ after Minimax Synchronization.
\end{theorem}
\begin{proof}
		Take any receiver $r \in \mathcal{R}$ and any sender $s \in \mathcal{S}$, let's consider the reorder delay before ($d$) and after ($d'$) synchronization between them.
		Use (\ref{proof:unchanged_FM}), we have
	\begin{equation*}\begin{split}
		 d &=\left(\FMGEN{\FM{r}}{r}\right)-\left(\FMGEN{s}{r}\right) \ge 0\\ d'&=|\left(\FMGEN{\FM{r}}{r}\right)-\left(O'_s - d_{sr}\right)| \ge 0
	\end{split}\end{equation*}
	
	If $O_s = O'_s$, apparently $d'=d$.
	
	If $O_s \neq O'_s$, from (\ref{proof:newOle}), we have
	$$\forall i \in \mathcal{S}, O_i \neq O'_i, \forall j \in \mathcal{R}:\FMGEN{\FM{j}}{j} > \FMGEN{i}{j}$$
	
	With (\ref{proof:FMnewOle}) and (\ref{proof:newOge}), we have $d' \le d \Rightarrow D' \le D$.
	
	In conclusion, the reorder delay between any pair of sender-receiver won't be worse after minimax synchronization, so will the reorder delay of overall system.
\end{proof}
\begin{rem}
	From Theorem~\ref{thm:someO_unchanged},~\ref{thm:allnewO_unchanged} and~\ref{thm:reorder_delay_decline}, we proved several important properties of our Minimax Synchronization:
	timestamp will converge in at most one rtt; reorder delay may decline (won't increase) after synchronization.
\end{rem}
\begin{note}
	If there are switches, the above proof still holds.
	Because the \textbf{min} and \textbf{max} are aggregate operations having associative property.
\end{note}

\begin{note}
	If consider delay variance, above proof shows the minimax synchronization can converge to a new balance in a single rtt.
\end{note}

\begin{note}
	We give but won't prove the drawbacks of choosing \textbf{max} or \textbf{avg} as $\texttt{BackFunc}$.
	If choose \textbf{max}, there exists conditions where logical timestamps may grow faster than physical clock.
	What worse is that the growth rate isn't constant, so we can't estimate link delay well without a steady clock.
	If choose \textbf{avg}, several faster timestamp may be held by many slower timestamp, which leads to a very long convergence time.
	This is unacceptable when merging two established clusters.
\end{note}
\fi