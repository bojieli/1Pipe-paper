\section{Appendix}
\newtheorem{theorem}{Theorem}[subsection]
\theoremstyle{definition}
\newtheorem{definition}{Definition}[subsection]
\newtheorem{property}{Property}[subsection]
\newtheorem*{assump}{Assumption}
\newtheorem*{notation}{Notation}
\newtheorem{lemma}[theorem]{Lemma}
\theoremstyle{remark}
\newtheorem*{rem}{Remark}
\newtheorem*{conclusion}{Conclusion}
\newtheorem*{note}{Note}
\subsection{Proof of hierarchical merge}
\label{appx:hierarchical_merge}
In Sec.~\ref{sec:ideal}, we have pointed out the property of barrier timestamp and how we merge barriers hierarchically.
Here we prove that the way we merge barriers does maintain its property.
Let's rewrite the property and formula:
\begin{property}\label{prop:barrier}
	Take any network link $L$, if a packet with barrier timestamp $B$ is on the $L$, then for any following packet carrying (barrier or data) timestamp $T$, there must be $B \le T$.
\end{property}
\begin{theorem}
	If we derive barriers following (\ref{equ:derive_barriers}), then the barrier timestamp $B$ determined for egress port $E$ still has the above property as long as the barrier from every ingress port does.
\end{theorem}
\begin{proof}
	We reuse notation $\mathcal{I}_E$ and $R_i$ defined in (\ref{equ:derive_barriers}).
	After barrier timestamp $B$ is sent through egress port $E$, all following packets are either data packets from a certain ingress port or barrier packets determined in the future.

	For a following data packet carrying timestamp $T$, let's assume it comes from ingress link $I \in \mathcal{I}$.
	Because all received barrier timestamps from $I$ have the Property~\ref{prop:barrier}, we have $B \le R_I \le T$.

	For a following barrier timestamp $T$, let's assume it is determined by newer register value $R'_I$.
	$$T:=\min\{R'_I|I \in \mathcal{I}_E\}$$
	
	Because of Property~\ref{prop:barrier}, $\forall I \in \mathcal{I}_E:R_I \le R'_I$, which implies $B \le T$.

	So, we can conclude that the newly determined barrier timestamp $B$ still has the Property~\ref{prop:barrier}.
\end{proof}

Besides, an endhost sender assigns barrier timestamps to be the same as data timestamps, which follows Property~\ref{prop:barrier} trivially, so hierarchical merge can maintain barrier's property properly.
\subsection{Proof of Minimax Synchronization}
\label{appx:minimax}

\begin{notation}
	Each sender $i\in\mathcal{S}$ has a logical timestamp $T_i$, which consists of physical clock $Phy$, physical clock skew $s_i$ and logical timestamp offset $O_i$ ($T_i = Phy+s_i+O_i$).
	Each receiver $j \in \mathcal{R}$ also has physical clock skew $s_j$.
\end{notation}\begin{notation}
	One-way delay from sender $i \in \mathcal{S}$ to receiver $j \in \mathcal{R}$ is $d_{ij}$.
	$d_{ij}$ is unmeasurable.
\end{notation}
\begin{notation}
	One-way delay from receiver $j \in \mathcal{R}$ to sender $i \in \mathcal{S}$ is $r_{ji}$.
	$r_{ji}$ is unmeasurable.
\end{notation}
\begin{notation}
	Round trip delay between sender $i \in \mathcal{S}$ and receiver $j \in \mathcal{R}$ is $RTT_{ij}( = RTT_{ji} = d_{ij} + r_{ji})$.
	$RTT_{ij}$ is measurable.
\end{notation}
\begin{note}
	A common endhost functions as both sender and receiver.
	So the delay between its two parts is $0$.
\end{note}
\begin{assump}
	There are no switches.
\end{assump}
\begin{assump}
	Senders and receivers are fully connected.
	The delay between physically unconnected sender and receiver is $+\infty$.
\end{assump}
\begin{assump}
	Delay is invariant during the synchronization.
\end{assump}
\begin{lemma}\label{proof:ts2offset}
	In the above model, logical timestamp offset is equivalent to logical timestamp.
\end{lemma}
\begin{proof}
	With linear time assumption, receiver $j$ will receive timestamp $(Phy+O_i-d_{ij}+s_i-s_j)$ from sender $i$ at physical time $Phy$.
	The receiver to sender direction is similar.
	We can count clock skew into link delay, which will make logical timestamp to only consist of physical time, timestamp offset and link delay.
	Besides, $Phy$ is the same for every sender and receiver, so we can treat logical timestamp offset $O_i$ equivalent to logical timestamp $T_i$.
\end{proof}
Now we prove several good mathematical properties assuming $\texttt{FwdFunc} =$ \textbf{max} and $\texttt{BackFunc} =$ \textbf{min}.
\theoremstyle{definition}
\begin{definition}{Let $O'_i$ be the new value of $O_i$}\label{proof:NextO_i}
\begin{equation*}
\begin{aligned}
O_i' &= \min\limits_{\forall j \in \mathcal{R}}\left[\max\limits_{\forall k \in \mathcal{S}}\left(O_k - d_{kj}\right) - r_{ji} + RTT_{ij}\right] \\
     &= \min\limits_{\forall j \in \mathcal{R}}\left[\max\limits_{\forall k \in \mathcal{S}}\left(O_k - d_{kj}\right) + d_{ij}\right] \\
     &=\min_{\forall  j \in \mathcal{R}}\left[\max_{\forall k \in \mathcal{S}}\left(O_k - d_{kj} + d_{ij}\right)\right]
\end{aligned}
\end{equation*}
\end{definition}
\newcommand{\FM}[1]{{\mathcal{M}_{#1}}}
\newcommand{\FMGEN}[2]{{O_{#1}-d_{{#1}{#2}}}}
\newcommand{\Fm}[1]{{m_{#1}}}
\newcommand{\FmGEN}[2]{\FMGEN{\FM{#1}}{#1}+ d_{{#2}{#1}}}
\newcommand{\NEXTT}[1]{{\FmGEN{\Fm{#1}}{#1}}}
\begin{definition} $\FM{j} \in \mathcal{S} (j \in \mathcal{R})$
\begin{equation} \label{proof:FMdef}
\forall j \in \mathcal{R}, \forall k \in \mathcal{S}:\FMGEN{\FM{j}}{j} \ge \FMGEN{k}{j}
\end{equation}
\end{definition}
\begin{definition} $\Fm{i} \in \mathcal{R} (i \in \mathcal{S})$
\begin{multline} \label{proof:Fmdef}
\forall i \in \mathcal{S}, \forall j \in \mathcal{R}:\\
(\FmGEN{\Fm{i}}{i}) \le (\FmGEN{j}{i})
\end{multline}
\end{definition}
\begin{note}
	The Definition~\ref{proof:NextO_i} can then be rewritten as
	\begin{equation}\label{proof:NextO_i2}
	O_i' \triangleq \NEXTT{i}
	\end{equation}
\end{note}
\begin{note}
	From (\ref{proof:Fmdef}) and (\ref{proof:NextO_i2}), we have
	\begin{equation}\label{proof:FMnewOle}
		\forall j \in \mathcal{R} : O'_i - d_{ij} \le \FMGEN{\FM{j}}{j}
	\end{equation}
	
	Which means
	\begin{equation}\label{proof:unchanged_FM}
		\forall j \in \mathcal{R}:O'_{\FM{j}} = O_{\FM{j}}
	\end{equation}
\end{note}
\begin{theorem}\label{thm:someO_unchanged}
	$\exists i \in \mathcal{S}:O'_i=O_i$
\end{theorem}
\begin{proof}
Let $k=i$ in (\ref{proof:FMdef}), we have
\begin{equation*}\begin{split}
\forall i \in \mathcal{S}, \forall j \in \mathcal{R} : \FMGEN{\FM{j}}{j} \ge O_i - d_{ij}\\
\Leftrightarrow\hspace{1em} \FmGEN{j}{i} \ge O_i
\end{split}\end{equation*}
With (\ref{proof:NextO_i2}), we have
\begin{equation} \label{proof:newOge}
\forall i \in \mathcal{S}: O'_i \ge O_i
\end{equation}
With (\ref{proof:Fmdef}) and (\ref{proof:NextO_i2}), we have
\begin{equation} \label{proof:newOle}
\forall i \in \{i \in \mathcal{S}| \exists j \in \mathcal{R}:\FM{j}=i\}:O'_i \le O_i
\end{equation}
With (\ref{proof:newOge}) and (\ref{proof:newOle}), we conclude
\begin{equation}\label{proof:O_i_unchanged}
\forall i \in \{i \in \mathcal{S}| \exists j \in \mathcal{R}:\FM{j}=i\}:O'_i = O_i
\end{equation}
\end{proof}
\begin{theorem}\label{thm:allnewO_unchanged}
	$\forall i \in \mathcal{S}: O''_i = O'_i$, where $O''_i$ is the new value of $O'_i$
\end{theorem}
\begin{proof}
From (\ref{proof:Fmdef}) and (\ref{proof:NextO_i2}), let $m_i = j$, we have

\begin{equation}\label{proof:newnewO1}
	\forall i,k \in \mathcal{S}: O'_k\le(\FmGEN{\Fm{i}}{k})
\end{equation}
Substitute the definition of $O_i'$ from (\ref{proof:NextO_i2}), we have
\begin{equation}\label{proof:newnewO2}
	\forall i,k \in \mathcal{S}: O'_k\le (O'_i + d_{k\Fm{i}} - d_{i\Fm{i}})
\end{equation}
From (\ref{proof:newnewO1}) and (\ref{proof:newnewO2}), we have
\begin{equation*}
	\forall i \in \mathcal{S}, \exists j \in \mathcal{R}, \forall k \in \mathcal{S}: (O'_k-d_{kj}+d_{ij}) \le O'_i
\end{equation*}
So by the Definition~\ref{proof:NextO_i}
\begin{equation} \label{proof:newnewOle}
O''_i = \min_{\forall  j \in \mathcal{R}}\left[\max_{\forall k \in \mathcal{S}}\left(O'_k - d_{kj} + d_{ij}\right)\right]\le O'_i
\end{equation}
Similar to (\ref{proof:newOge}), we can prove
\begin{equation}\label{proof:newnewOge}
	\forall i \in \mathcal{S}:O''_i \ge O'_i
\end{equation}
From (\ref{proof:newnewOle}) and (\ref{proof:newnewOge}), we have $\forall i \in \mathcal{S}: O''_i=O'_i$
\end{proof}
\begin{definition}
	The reorder delay $D$ of overall system is the maximum timestamp skew among all receivers.
	\begin{equation*}
		D:=\max_{\substack{\forall j \in \mathcal{R}\\\forall k \in \mathcal{S}}}\left[\left(\FMGEN{\FM{j}}{j}\right)-\left(\FMGEN{k}{j}\right)\right]
	\end{equation*}
\end{definition}
\begin{theorem}\label{thm:reorder_delay_decline}
	With arbitrary initial $O_i(i\in \mathcal{S})$, the initial reorder delay $D$ $\ge$ the reorder delay $D'$ after Minimax Synchronization.
\end{theorem}
\begin{proof}
		Take any receiver $r \in \mathcal{R}$ and any sender $s \in \mathcal{S}$, let's consider the reorder delay before ($d$) and after ($d'$) synchronization between them.
		Use (\ref{proof:unchanged_FM}), we have
	\begin{equation*}\begin{split}
		 d &=\left(\FMGEN{\FM{r}}{r}\right)-\left(\FMGEN{s}{r}\right) \ge 0\\ d'&=|\left(\FMGEN{\FM{r}}{r}\right)-\left(O'_s - d_{sr}\right)| \ge 0
	\end{split}\end{equation*}
	
	If $O_s = O'_s$, apparently $d'=d$.
	
	If $O_s \neq O'_s$, from (\ref{proof:newOle}), we have
	$$\forall i \in \mathcal{S}, O_i \neq O'_i, \forall j \in \mathcal{R}:\FMGEN{\FM{j}}{j} > \FMGEN{i}{j}$$
	
	With (\ref{proof:FMnewOle}) and (\ref{proof:newOge}), we have $d' \le d \Rightarrow D' \le D$.
	
	In conclusion, the reorder delay between any pair of sender-receiver won't be worse after minimax synchronization, so will the reorder delay of overall system.
\end{proof}
\begin{rem}
	From Theorem~\ref{thm:someO_unchanged},~\ref{thm:allnewO_unchanged} and~\ref{thm:reorder_delay_decline}, we proved several important properties of our Minimax Synchronization:
	timestamp will converge in at most one rtt; reorder delay may decline (won't increase) after synchronization.
\end{rem}
\begin{note}
	If there are switches, the above proof still holds.
	Because the \textbf{min} and \textbf{max} are aggregate operations having associative property.
	So switches introduced in Algorithm~\ref{alg:minimax} don't affect the results.
\end{note}

\begin{note}
	If consider delay variance, above proof shows the minimax synchronization can converge to a new balance in a single rtt.
\end{note}

\begin{note}
	We give but won't prove the drawbacks of choosing \textbf{max} or \textbf{avg} as $\texttt{BackFunc}$.
	If choose \textbf{max}, there exists conditions where logical timestamps may grow faster than physical clock.
	What worse is that the growth rate isn't constant, so we can't compensate link delay well without a steady clock.
	If choose \textbf{avg}, several faster timestamp may be held by many slower timestamp for a very long time.
	This is unacceptable when merging two established clusters.
\end{note}