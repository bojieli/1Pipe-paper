\section{Related Work}
\label{sec:related}

%\RED{NetChain. Section 4.3 of the paper shows that comparing sequence numbers for each item (network port) is implementable in network switches. Furthermore, the use of sequence number to validate serialization order implies that reordering packets buffered in a network switch is not possible. This agrees with the design principle of TOMS.}

\textbf{Total-order multicast.}
Efficiency and scalability is considered a fundamental trade-off for total-order multicast~\cite{defago2004total}.
For efficiency, a centralized algorithm is often used, \textit{e.g.}, sequencers~\cite{eris} or token passed among senders or receivers~\cite{rajagopalan1989token,kim1997total,ekwall2004token}.
Scalability often leads to a distributed algorithm, \textit{e.g.}, aggregate history during message delivery~\cite{chandra1996unreliable} or run a consensus protocol among receivers~\cite{lamport1978time}.
Critics of total order communication~\cite{cheriton1994understanding} blame the lack of atomicity, limited scalability and violation of end-to-end principle.
\sys provides atomicity with message scattering, achieves scalability with in-network computation and incurs little overhead.

\sys is a variant of tree-based atomic broadcast~\cite{rodrigues1998scalable}, which is used in multi-cores~\cite{kaestle2016machine} and sensor networks~\cite{chakraborty2011reliable}.
To merge multiple ordered streams of packets into one ordered stream, a classic algorithm is \textit{deterministic merge}~\cite{hadzilacos1994modular, aguilera2000efficient}.
%It works like a merge sort: Each step the network switch compares the heads of input streams and output the head packet with minimum timestamp. %Reordering packets on the switch introduces less reordering delay compared to reordering on the receivers (as shown in Figure~\ref{fig:barrier}), because the switch sees messages more frequently~\cite{aguilera2000efficient}.
%However, when the timestamps are not perfectly synchronized, the switch needs to wait and buffer other streams.
%The maximum time of waiting equals the network delay variance plus the clock skew among senders. According to Sec.\ref{sec:goals}, the delay variance may be milliseconds, therefore for a 40~Gbps network, the switch needs a 5~MB buffer per port to hold out-of-order packets. 
However, per-port buffer size in commodity switches is much smaller than bandwidth-delay product~\cite{bai2017congestion}.
In addition, commodity switches do not support scheduling queues according to per-packet metadata~\cite{sivaraman2016programmable,jin2018netchain}.
In light of these limitations, we propose the principle of separating ordering information from message forwarding.

\textbf{Hardware accelerated concurrency control.}
%There is rich literature in achieving strong consistency efficiently in distributed systems.%~\cite{lloyd2011don,lloyd2013stronger,mu2014extracting,zhang2016operation,lu2015existential,ajoux2015challenges,mu2016consolidating,lu2016snow,kallman2008h,zhang2015building}.
For lock-based concurrency control, most of the recent works~\cite{dragojevic2014farm,kalia2016fasst,kaminsky2016design,dragojevic2015no} use RDMA to offload network processing and lock handling to NICs.
To reduce transaction abort rate in timestamp-based OCC, the key idea is to process transactions in monotonic timestamp order.
Mostly-Ordered Multicast~\cite{ports2015designing}, NOPaxos~\cite{li2016just}, Eris~\cite{eris} and NetChain~\cite{jin2018netchain} co-design distributed systems with programmable network switches to ensure total ordering of message timestamps. 
However, these designs use a network switch as centralized sequencer, which has limited scalability.

%\textbf{In-network computation.}
%In-network computation flourishes with the trend of programmable switches~\cite{lu2011serverswitch,tofino,bosshart2013forwarding} and NICs~\cite{kaufmann2016high,clicknp}.
%Programmable switches can cache frequently accessed data in key-value stores~\cite{li2016fast,netcache-sosp17,kv-direct} and network infrastructure services~\cite{fayazbakhsh2013less,liu2017incbricks,miao2017silkroad}.
%The wide visibility of programmable switches make them a better place to implement consensus protocols~\cite{dang2016network,dang2016paxos,dang2015netpaxos}.

\textbf{Bounded delay.}
A network with bounded delay can improve reordering delay of \sys.
Achieving bounded delay requires the queue length to be bounded and the network to be lossless.
Recent years see progress in lossless data center networks~\cite{calder2013don,cheng2014catch,handley2017re} and fast detection of packet losses~\cite{li2016lossradar}.
Bounded queuing delay is also enabled by centralized controller~\cite{perry2015fastpass} or end-to-end flow control~\cite{cho2017credit}.




\iffalse


Spanner: Google’s globally distributed database (OSDI\'12)~\cite{corbett2013spanner},
Linearizabiliy: A correctness condition for concurrent objects (ACM Trans, 1990),~\cite{herlihy1990linearizability}
Low overhead concurrency control for partitioned main memory databases (SIGMOD\'10),~\cite{jones2010low}
H-Store: a highperformance,
distributed main memory transaction processing
system (VLDB\'08),~\cite{kallman2008h}
MDCC: multi-data center consistency (Eurosys\'13),~\cite{kraska2013mdcc}
Calvin: Fast distributed transactions for
partitioned database systems~\cite{thomson2012calvin}
Fast Distributed Transactions and Strongly Consistent Replication
for OLTP Database Systems (Calvin'14),~\cite{thomson2014fast}
Transaction Processing Performance Council. TPC
Benchmark C.~\cite{council2005transaction}

Receiver time synchronization with MIN:
Fine-Grained Network Time Synchronization using Reference
Broadcasts, OSDI'02.~\cite{elson2002fine}
Virtual time, 1985.~\cite{jefferson1985virtual}


Zookeeper: A simple totally ordered broadcast protocol

Raft

The End of an Architectural Era
(It’s Time for a Complete Rewrite)

OLTP through the looking glass, and what we found there

The VoltDB Main Memory DBMS

Viewstamped Replication [19] and Raft [20], ZAB [16], and Multi-Paxos [3]

In modern distributed databases, consensus is often used as the basis for replication, to ensure replicas apply updates in the same order, an instance of state-machine replication (see Schneider’s tutorial Schneider, F.B. Implementing fault-tolerant services using the state machine approach: A tutorial).

We solve locking (2PL): Jim Gray, Raymond A. Lorie, Gianfranco R. Putzolu, Irving L. Traiger. Granularity of Locks and Degrees of Consistency in a Shared Data Base. , IBM, September, 1975.

Stonebraker: My current best guess is that nobody will use traditional two phase locking. Techniques based on timestamp ordering or multiple versions are likely to prevail. The third paper in this section discusses Hekaton, which implements a state-of-the art MVCC scheme.

Cristian Diaconu, Craig Freedman, Erik Ismert, Per-Ake Larson, Pravin Mittal, Ryan Stonecipher, Nitin Verma, Mike Zwilling. Hekaton: SQL Server's Memory-optimized OLTP Engine. SIGMOD, 2013.



Orthogonal to 2PC (Two phase commit):
Lampson, B. and Sturgis, H. Crash recovery in a distributed data storage system. Xerox PARC Technical Report. 1979.

On Interprocess Communication, Lamport, regular register

Linearizability: A Correctness Condition for
Concurrent Objects, Jennette Wing

The serializability of concurrent database updates, 1979

Useless Actions Make a Difference:
Strict Serializability of Database Updates

Berkeley algorithm (average):
The accuracy of the clock synchronization achieved by TEMPO in Berkeley UNIX 4.3BSD

SRM (Scalable Reliable Multicast):
A Reliable Multicast Framework for Light-weight Sessions
and Application Level Framing

Scalable multicast:
Hierarchical feedback 

The Implementation of Reliable Distributed Multiprocess Systems

control~\cite{tanenbaum2007distributed}


Physical clock synchronization: Fault-Tolerant Clock Synchronization, 
Optimal clock synchronization

Sync to fastest clock: Clock synchronization algorithm for address independent networks (patent),
\fi
