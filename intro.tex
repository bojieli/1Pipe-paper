\section{Introduction}
\label{sec:intro}

\iffalse
\begin{figure}[t]
\centering
	\subfloat[Unordered communication.\label{fig:lightcone-traditional}]
	{\includegraphics[width=.23\textwidth,page=1]{images/cropped_lightcone.pdf}}
	\subfloat[Totally ordered communication.\label{fig:lightcone-toms}]
	{\includegraphics[width=.23\textwidth,page=2]{images/cropped_lightcone.pdf}}
	\caption{Light cone of observed events by a receiver.}
	\label{fig:lightcone}
    \vspace{-15pt}
\end{figure}
\fi

\RED{Need to have a concrete ordering example (causal ordering). TOMS can be a building block for both strongly consistent and eventually consistent systems.}
Today’s datacenters host a variety of distributed systems to provide service for global users.
The ability to order events in a distributed system is essential for the correctness of many distributed protocols~\cite{lamport1978time,chandy1985distributed}.
Traditionally, in distributed systems where network is lossy and has arbitrary message delays, a receiver is never sure to have received all messages before a certain event.
%as shown in Figure~\ref{fig:lightcone-traditional}. 
Totally ordered communication provides an abstraction with a new view of space-time, 
where different receivers process messages from senders in a consistent order.% (Figure~\ref{fig:lightcone-toms}).
In a totally ordered communication system, messages are totally ordered with \textit{event timestamps}, and each receiver processes all messages sent to it based on timestamp order. This can simplify and accelerate many distributed applications, \textit{e.g.}, transactional key-value stores~\cite{ports2015designing, eris}, total store ordering~\cite{sewell2010x86} in distributed shared memory, fault-tolerant consensus~\cite{li2016just}, mutual exclusion~\cite{lamport1978time}, state machine replication~\cite{lamport1978time,lamport1978implementation} and distributed snapshots~\cite{chandy1985distributed}.

%Totally ordered message scattering provides an abstraction with a new view of space-time, where different receivers process messages scattered from senders in a consistent order (Figure~\ref{fig:light-cone} (b)). In a totally ordered message scattering system, messages are totally ordered with \textit{event timestamps}, and each receiver receives all messages scattered to it in the timestamp order. This can simplify and accelerate many distributed applications, \textit{e.g.}, transactional key-value stores~\cite{ports2015designing, eris}, total store ordering in distributed shared memory~\cite{}, fault-tolerant consensus~\cite{li2016just}, mutual exclusion~\cite{lamport1978time}, state machine replication~\cite{lamport1978time,lamport1978implementation} and distributed snapshots~\cite{chandy1985distributed}.

%The ordering of events is fundamental to distributed systems.
%The ability to \textit{total order} events in a distributed system, \textit{i.e.}, all nodes observe a consistent ordering of all events, can lead to simple and efficient implementation of a wide range of distributed applications, \textit{e.g.}, transaction processing~\cite{ports2015designing, eris}, transactional key-value stores~\cite{ports2015designing, eris}, fault-tolerant consensus~\cite{li2016just}, mutual exclusion~\cite{lamport1978time}, state machine replication~\cite{lamport1978time,lamport1978implementation} and distributed snapshots~\cite{chandy1985distributed}.
%A total ordering of events indicates that each event can be tagged with a \RED{\textit{logical timestamp}}, and each node observes events in increasing timestamp order (break ties by origin node ID).

%In a networked system, effects of an event propagate via network messages. This means to \textit{scatter} messages from the origin node to a set of receiver nodes. For communication efficiency, an event may not need to be observed by all nodes, and different nodes may receive different messages originated from a same event, so generally it is a \textit{scattering} instead of a \textit{broadcast} or \textit{multicast}. Messages are tagged with the timestamp of the origin event. A node observes an event by receiving the corresponding message. Consequently, from a receiver's perspective, \textit{total ordering of input message timestamps implies total ordering of its observed events}.

Since the dawn of distributed system research~\cite{lamport1978time}, many efforts have been made to achieve totally ordered communication, especially broadcast and multicast~\cite{defago2004total}. In this paper, we consider a more general form of communication, namely \emph{message scattering}. Message scattering is a common communication pattern where a sender sends (potentially different) messages to multiple receivers. In contrast, identical messages are sent in multicast and broadcast. In a totally ordered message scattering system, different receivers will process scattered messages in a consistent order. If $M_1$ is processed before $M_2$ at receiver $A$, then the same processing order should be kept at receiver $B$, \textit{i.e.}, $B$ will process $M'_1$ before $M'_2$ ($M_1$ and $M'_1$ belong to the same scattering, though the message content could be different).

Most totally ordered multicast algorithms can be used to implement totally ordered message scattering with some modifications. However, existing solutions suffer from scalability or efficiency limitations. One line of work leverages logically centralized coordination, \textit{e.g.}, centralized sequencers~\cite{eris}, or tokens to be passed among senders and receivers~\cite{rajagopalan1989token,kim1997total,ekwall2004token}. As a result, the system throughput is hard to scale. Another line of work uses fully distributed coordination, \textit{e.g.}, add a consensus round among receivers before they start to process messages~\cite{lamport1978time,chandra1996unreliable}. This causes extra network communication overhead and delay, thus degrading the system efficiency.

Realizing the above limitations, we seek a solution that can provide efficient and scalable totally ordered message scattering for distributed systems. In particular, we focus on data center context in this paper, since many of today's high performance distributed systems are deployed in data centers. Unlike Internet, the data center network has a number of desirable properties such as regular topologies~\cite{leiserson1985fat,greenberg2009vl2} and single administrative domain. Furthermore, programmable switches start to flourish in data center networks, providing more flexible packet processing. Recently, there has been a trend to co-design distributed systems with underlying data center networks~\cite{eris,netcache-sosp17,dang2016paxos}. In this paper, we follow this trend and propose \sys to achieve efficient and scalable totally ordered message scattering in \emph{programmable data center networks.}   
 
%Since the dawn of distributed systems research~\cite{lamport1978time}, there has been considerable amount of literature on total-ordering events and input messages using \textit{total order broadcast}~\cite{defago2004total}.
%Most total-order broadcast algorithms can be modified to implement total-order message scattering, but they have scalability or efficiency limitations.
%One line of work uses logically centralized coordination, \textit{e.g.}, using one or more centralized sequencers~\cite{eris}, or have a token to be passed among senders or receivers~\cite{}. The throughput of such systems is hard to scale.
%Another line of work uses fully distributed coordination, \textit{e.g.}, add a consensus round among receivers after they receive the messages~\cite{}. The network communication overhead and additional consensus delay are not negligible.


%Different from the traditional assumption that the network is an unreliable blackbox, a datacenter network has a regular topology and a single administrative domain.
%Recent years, as programmable network switches flourish, there has been a trend to co-design the network with distributed systems~\cite{}. By offloading end-host computation to network switches, key-value stores can be load balanced better~\cite{} and distributed coordination can be simple and fast~\cite{eris}.
%This paper follows this trend and shows that in programmable datacenter networks, total-order message scattering can achieve high performance, be scalable both inside and across data centers, without changing existing network infrastructure.

At its core, \sys separates the bookkeeping of order information from message forwarding. On the data plane, \sys attaches each message a timestamp at the sender side, forwards them as usual in the network, and buffers them at the receiver side. On the control plane, the switch aggregates timestamp information of all messages to derive the \textit{barrier} for each receiver. The barrier is essentially the timestamp \textit{lower bound} of all future arrival packets. With this information, the receiver reorders messages below the barrier and deliver them to applications. \sys is scalable as it distributes the work to each switch and the end host.

To ensure \sys's efficiency, we still need to solve three main challenges: \textit{How to derive timestamp barriers? How to assign event timestamps? How to handle packet losses and node failures?} 

To solve the first challenge, we first generalize timestamp barriers from end hosts to every link in the network. Each switch keeps per-link barrier information and updates it for each packet. When some hosts or links are temporarily idle, we proactively generate some beacons carrying barrier information. (Sec.\ref{sec:beacon}). We further merge barriers hierarchically at switches to reduce the amount of beacon traffic (Sec.\ref{sec:ideal}). 

For the second challenge, physical clock synchronization seems to be a straightforward solution. However, it does not account for the delay difference incurred by OS and links, resulting in high reordering delay. For example, two messages are transmitted from two senders to the same receiver at the same physical time. However, they arrive at different times due to their different network path lengths. As a result, the former one must wait for later one to be processed together. To minimize the impact, we propose \textit{minimax clock synchronization} (Sec.\ref{sec:sync}) to synchronize \textit{logical clocks} on each node and assign timestamps to events according to the logical clocks.

\RED{Need to discuss atomic message scattering based on 2PC, improvement (fast commit when no loss), node failure.}
For the third challenge, we are aware that data centers are typically well engineered and have very low packet loss rates~\cite{ports2015designing}. We detect packet loss by packet drop counters in network switches, and rely on the end hosts to retransmit lost packets (Sec.\ref{sec:lossy}). With an additional round-trip delay, ACK barrier provides reliable total order message scattering.

%In Sec.\ref{sec:lossy}, we add reliability to total-order message scattering in lossy networks, with overhead of one round-trip delay. 


%A simple existing approach to totally order messages to the end hosts is called \textit{deterministic merge}~\cite{hadzilacos1994modular, aguilera2000efficient}, which serialize network messages at each network switch. However, commodity switches do not have enough buffer capacity and the programmability to \textit{merge sort} ingress streams (cite PIFO).

%In this work, we propose a different approach to achieve totally ordered scattering. The messages are forwarded as normal in network switches and buffered in end-host receivers. Network switches aggregate timestamp \textit{barrier} information to indicate the \textit{lower bound} of all future timestamps that can be received by an end host. The end hosts reorder messages below the timestamp barrier and deliver them to applications.
%Two challenges arise: \textit{How to derive timestamp barriers? How to assign event timestamps?}

%To derive timestamp barriers, we generalize timestamp barriers from end hosts to every link in the network. A barrier packet on a link indicates the lower bound of the timestamps of all packet that can arrive on the link after the particular barrier packet. To avoid exponential number of barrier packets, barriers are merged hierarchically on network switches (Sec.\ref{sec:ideal}).
%When some hosts or network links are temporarily idle, beacons are sent (Sec.\ref{sec:beacon}).
%Because packet loss in datacenter networks are rare~\cite{ports2015designing}, we could follow the end-to-end principle in system design~\cite{saltzer1984end} and leave packet loss detection and recovery to applications, as in~\cite{ports2015designing,li2016just}.
%In Sec.\ref{sec:lossy}, we add reliability to total-order message scattering in lossy networks, with overhead of one round-trip delay.

%Solving the first challenge ensures correctness of our design. The second challenge, namely assignment of event timestamps, affects efficiency of the design.
%If an event source assigns its messages with very high timestamps, on the receiver end, its messages need to wait in the buffer for a long time for the packets with lower timestamps from other event sources to arrive.
%Our goal is to minimize \textit{reordering delay} from receiving the message to delivering it to the application.
%Physical clock synchronization does not account for the difference in delays incured by OS network stacks and network links.
%In Sec.\ref{sec:sync}, we propose \textit{minimax clock synchronization} to synchronize logical clocks on each node and assign timestamps to events according to the logical clocks, so that messages originated from distant nodes with adjacent timestamps arrive at receivers as simultaneously as possible.

We design and implement \sys in three scenarios with different programming capabilities of network switches. Sec.\ref{sec:p4} assumes stateful per-packet processing with data-plane programmable switches. Sec.\ref{sec:commodity} assumes commodity switches with a programmable CPU, which can process control packets but not each and every data packets. In case switch CPUs do not have enough processing capacity, Sec.\ref{sec:end-host} uses end hosts for control plane processing.

\RED{Update evaluation scenarios.}
\sys achieves low reordering delay and low network overhead and CPU processing overhead in both small and large system scales. \sys is also fault tolerant and supports incremental deployment, as new hosts, switches and links can join the system within only one RTT. As a case study, Sec.~\ref{sec:application} shows that \sys achieves 6.4 million transactions per server and 40~$\mu$s latency for single round-trip transactions in YCSB+T~\cite{dey2014ycsbt} transactional key-value store, which is 10 times more efficient than other systems and close to the performance of the non-transactional system.

